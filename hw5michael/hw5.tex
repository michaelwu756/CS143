\documentclass[12pt]{article}
\usepackage{amsmath}
\begin{document}
\title{Computer Science 143, Homework 5}
\date{June 8th, 2018}
\author{Michael Wu\\UID: 404751542}
\maketitle

\section*{Problem 1}

\paragraph{a)}

Yes. Each transaction does not overlap with any other transaction.

\paragraph{b)}

Yes it is conflict serializable, as it is already serial. The order of the transactions are currently \(T_3\rightarrow T_1\rightarrow T_2\rightarrow T_4\).
Note that this schedule is conflict equivalent to \(T_3\rightarrow T_1\rightarrow T_4\rightarrow T_2\).

\section*{Problem 2}

\paragraph{a)}

We have the following execution schedules.
\begin{gather*}
        T_1\rightarrow T_2\rightarrow T_3\\
        T_1\rightarrow T_3\rightarrow T_2\\
        T_2\rightarrow T_1\rightarrow T_3\\
        T_2\rightarrow T_3\rightarrow T_1\\
        T_3\rightarrow T_1\rightarrow T_2\\
        T_3\rightarrow T_2\rightarrow T_1
\end{gather*}
These yield the following values in order: 30, 30, 42, 36, 25, 37.

\paragraph{b)}

Since an update statement is both a read and a write, the transaction \(T_2\) will grab a lock on both rows in the table. It then releases it at the
end of the transaction. So \(T_3\) cannot write in between \(T_2\), and \(T_2\) and \(T_3\) must be serializable with respect to each other.
However, the transaction \(T_1\) can read at any time. Let \(T_2=AB\) and \(T_3=CD\), where \(A\), \(B\), \(C\), and \(D\) are
the atomic statements that make up \(T_2\) and \(T_3\). Then we have the following possible execution schedules.
\[
        \begin{array}{c|c}
                \text{Schedule} & \text{Output}\\
                \hline
                T_1ABCD & 30\\
                AT_1BCD & 32\\
                ABT_1CD & 42\\
                ABCT_1D & 52\\
                ABCDT_1 & 36\\
                T_1CDAB & 30\\
                CT_1DAB & 40\\
                CDT_1AB & 25\\
                CDAT_1B & 27\\
                CDABT_1 & 37
        \end{array}
\]

\end{document}