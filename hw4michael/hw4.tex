\documentclass[12pt]{article}
\usepackage{amsmath}
\usepackage{amssymb}
\begin{document}
\title{Computer Science 143, Homework 4}
\date{June 1st, 2018}
\author{Michael Wu\\UID: 404751542}
\maketitle

\section*{Part 1}

\subsection*{Problem 1}

Yes, this is a lossless decomposition. Since \(A\rightarrow BC\) and \(B\rightarrow D\), we have that \(A\rightarrow BCD\). Then
since \(CD\rightarrow E\), we have that \(A\rightarrow BCDE\). So \(A\rightarrow ADE\), and \(A\) is a superkey for \(R_2\). Since
\(R_1 \cap R_2 = \{A\}\), we have \(R_1 \cap R_2 \rightarrow R_2\).

\subsection*{Problem 2}

\[\{A\rightarrow B, C\rightarrow B, C\rightarrow A\}\]

\subsection*{Problem 3}

\paragraph{a)}

Yes. We have that \(E\rightarrow A\) and \(A\rightarrow BC\), so \(E\rightarrow ABC\). Then we have \(B\rightarrow D\), so \(E\rightarrow ABCD\).
Along with the trivial functional dependency \(E\rightarrow E\), we get \(E\rightarrow ABCDE\) or \(E\rightarrow R\). Thus \(E\) is a key for \(R\).

\paragraph{b)}

Yes. We have the trivial functional dependency \(BC\rightarrow BC\) and \(B\rightarrow D\), so \(BC\rightarrow BCD\). Then we have \(CD\rightarrow E\),
so \(BC\rightarrow BCDE\). Finally we have \(E\rightarrow A\), so \(BC\rightarrow ABCDE\), or \(BC\rightarrow R\). Thus \(BC\) is a key for \(R\).

\subsection*{Problem 4}

This is not in BCNF. Consider the functional dependency \(B\rightarrow D\). The dependency \(B\rightarrow F\) cannot be derived from the closure of the
set of funtional dependencies, so \(B\) is not a key for \(R\). Thus there exists some nontrivial functional dependency in \(R\) where the left hand side
is not a key for \(R\). If it was in BCNF, this could not happen. To normalize it into BCNF, decompose it into the relations \(R_1(A,F)\), \(R_2(A,B,C)\),
\(R_3(C,E)\), and \(R_4(B,D)\).

\subsection*{Problem 5}

\begin{multline*}
        \{(a,b_1,c_1,d_2),(a,b_1,c_1,d_3),(a,b_2,c_2,d_1),\\
        (a,b_2,c_2,d_3),(a,b_3,c_3,d_1),(a,b_3,c_3,d_2)\}
\end{multline*}

\subsection*{Problem 6}

No it is not in 4NF. We have a nontrivial multivalue dependency \(A\twoheadrightarrow B\) where \(A\) is not a key, which violates the
conditions of 4NF. To normalize it into 4NF, decompose it into the relations \(R_1(A,D,F)\), \(R_2(A,B)\), \(R_3(A,C)\), and \(R_4(A,E)\).

\section*{Part 2}

\subsection*{Problem 1}

\subsection*{Problem 2}

\section*{Part 3}

\subsection*{Problem 1}

\subsection*{Problem 2}

\end{document}