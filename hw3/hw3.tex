\documentclass[12pt]{article}
\usepackage{amsmath}
\usepackage{graphicx}
\begin{document}
\title{Computer Science 143, Homework 3}
\date{May 22nd, 2018}
\author{Michael Wu\\UID: 404751542}
\maketitle

\section*{Part 1}

I found all the distinct company values and was able to generate the following query to convert from a long format to a wide format.
This gave a result with 47 columns and 67 rows.
{\footnotesize
\begin{verbatim}
SELECT company,
  MAX(IF(value='bonded-by-product', 1, 0)) AS bonded_by_product,
  MAX(IF(value='project-ownership', 1, 0)) AS project_ownership,
  MAX(IF(value='friends-outside-work', 1, 0)) AS friends_outside_work,
  MAX(IF(value='feedback', 1, 0)) AS feedback,
  MAX(IF(value='impressive-teammates', 1, 0)) AS impressive_teammates,
  MAX(IF(value='personal-growth', 1, 0)) AS personal_growth,
  MAX(IF(value='fast-paced', 1, 0)) AS fast_paced,
  MAX(IF(value='lunch-together', 1, 0)) AS lunch_together,
  MAX(IF(value='remote-ok', 1, 0)) AS remote_ok,
  MAX(IF(value='customer-first', 1, 0)) AS customer_first,
  MAX(IF(value='many-hats', 1, 0)) AS many_hats,
  MAX(IF(value='quality-code', 1, 0)) AS quality_code,
  MAX(IF(value='open-communication', 1, 0)) AS open_communication,
  MAX(IF(value='internal-promotion', 1, 0)) AS internal_promotion,
  MAX(IF(value='retention', 1, 0)) AS retention,
  MAX(IF(value='product-driven', 1, 0)) AS product_driven,
  MAX(IF(value='worklife-balance', 1, 0)) AS worklife_balance,
  MAX(IF(value='light-meetings', 1, 0)) AS light_meetings,
  MAX(IF(value='flex-hours', 1, 0)) AS flex_hours,
  MAX(IF(value='inclusive', 1, 0)) AS inclusive,
  MAX(IF(value='psychologically-safe', 1, 0)) AS psychologically_safe,
  MAX(IF(value='diverse-team', 1, 0)) AS diverse_team,
  MAX(IF(value='eq-iq', 1, 0)) AS eq_iq,
  MAX(IF(value='parents', 1, 0)) AS parents,
  MAX(IF(value='open-source', 1, 0)) AS open_source,
  MAX(IF(value='continuous-delivery', 1, 0)) AS continuous_delivery,
  MAX(IF(value='engages-community', 1, 0)) AS engages_community,
  MAX(IF(value='cross-dep', 1, 0)) AS cross_dep,
  MAX(IF(value='safe-env', 1, 0)) AS safe_env,
  MAX(IF(value='rapid-growth', 1, 0)) AS rapid_growth,
  MAX(IF(value='new-tech', 1, 0)) AS new_tech,
  MAX(IF(value='creative-innovative', 1, 0)) AS creative_innovative,
  MAX(IF(value='data-driven', 1, 0)) AS data_driven,
  MAX(IF(value='flat-organization', 1, 0)) AS flat_organization,
  MAX(IF(value='engineering-driven', 1, 0)) AS engineering_driven,
  MAX(IF(value='agile-dev', 1, 0)) AS agile_dev,
  MAX(IF(value='pair-programs', 1, 0)) AS pair_programs,
  MAX(IF(value='team-oriented', 1, 0)) AS team_oriented,
  MAX(IF(value='internal-mobility', 1, 0)) AS internal_mobility,
  MAX(IF(value='physical-wellness', 1, 0)) AS physical_wellness,
  MAX(IF(value='benefit-company', 1, 0)) AS benefit_company,
  MAX(IF(value='interns', 1, 0)) AS interns,
  MAX(IF(value='junior-devs', 1, 0)) AS junior_devs,
  MAX(IF(value='risk-taking', 1, 0)) AS risk_taking,
  MAX(IF(value='good-beer', 1, 0)) AS good_beer,
  MAX(IF(value='office-layout', 1, 0)) AS office_layout
FROM keyvalues
GROUP BY company;
\end{verbatim}
}

\section*{Part 2}

\begin{enumerate}
        \item No Subquery with \texttt{DISTINCT} \hfill
        \begin{center}
                \hspace*{-7cm}
                \resizebox{1.3\textwidth}{!}{%
                \begin{tabular}{c|c|c|c|c|c|c|c|c|c|c|c}
                        id & select\_type & table & partitions & type & possible\_keys & key & key\_len & ref & rows & filtered & extra\\
                        \hline
                        1 & SIMPLE & caltrans & NULL & ALL  & PRIMARY & NULL & NULL & NULL & 103056 & 20.99 & Using where; Using temporary; Using filesort
                \end{tabular}
                }
                \hspace*{-7cm}
        \end{center}
        \item \texttt{SELECT} within a \texttt{SELECT} \hfill
        \begin{center}
                \hspace*{-7cm}
                \resizebox{1.3\textwidth}{!}{
                \begin{tabular}{c|c|c|c|c|c|c|c|c|c|c|c}
                        id & select\_type & table & partitions & type & possible\_keys & key & key\_len & ref & rows & filtered & extra \\
                        \hline
                        1 & PRIMARY & \textless derived2\textgreater & NULL & ALL & NULL & NULL & NULL & NULL & 2402 & 100.00 & Using temporary; Using filesort \\
                        2 & DERIVED & caltrans & NULL & ALL & PRIMARY & NULL & NULL & NULL & 103056 & 2.33 & Using where; Using temporary; Using filesort
                \end{tabular}
                }
                \hspace*{-7cm}
        \end{center}
        \item \texttt{JOIN} as a Filter \hfill
        \begin{center}
                \hspace*{-7cm}
                \resizebox{1.3\textwidth}{!}{
                \begin{tabular}{c|c|c|c|c|c|c|c|c|c|c|c}
                        id & select\_type & table & partitions & type & possible\_keys & key & key\_len & ref & rows & filtered & extra \\
                        \hline
                        1 & PRIMARY & \textless derived2\textgreater & NULL & ALL & NULL & NULL & NULL & NULL & 24047 & 100.00 & Using temporary; Using filesort \\
                        2 & DERIVED & c & NULL & ALL & PRIMARY & NULL & NULL & NULL & 103056 & 2.33 & Using where; Using temporary; Using filesort \\
                        2 & DERIVED & \textless derived3\textgreater & NULL & ref & \textless auto\_key0\textgreater & \textless auto\_key0\textgreater & 8 & hw2.c.highway & 10 & 100.00 & NULL \\
                        3 & DERIVED & caltrans & NULL & ALL & PRIMARY & NULL & NULL & NULL & 103056 & 2.33 & Using where; Using temporary
                \end{tabular}
                }
                \hspace*{-7cm}
        \end{center}
        \item Using an \texttt{IN} Subquery as a Filter\hfill
        \begin{center}
                \hspace*{-7cm}
                \resizebox{1.3\textwidth}{!}{
                \begin{tabular}{c|c|c|c|c|c|c|c|c|c|c|c}
                        id & select\_type & table & partitions & type & possible\_keys & key & key\_len & ref & rows & filtered & extra \\
                        \hline
                        1 & PRIMARY & \textless derived2\textgreater & NULL & ALL & NULL & NULL & NULL & NULL & 2402 & 100.00 & Using temporary; Using filesort \\
                        2 & DERIVED & c & NULL & ALL & PRIMARY & NULL & NULL & NULL & 103056 & 2.33 & Using where; Using temporary; Using filesort \\
                        2 & DERIVED & \textless subquery3\textgreater & NULL & eq\_ref & \textless auto\_key\textgreater & \textless auto\_key\textgreater & 265 & hw2.c.highway,hw2.c.area & 1 & 100.00 & NULL \\
                        3 & MATERIALIZED & caltrans & NULL & ALL & NULL & NULL & NULL & NULL & 103056 & 2.33 & Using where
                \end{tabular}
                }
                \hspace*{-7cm}
        \end{center}
        \item Formal Left Semijoin \hfill
        \begin{center}
                \hspace*{-7cm}
                \resizebox{1.3\textwidth}{!}{
                \begin{tabular}{c|c|c|c|c|c|c|c|c|c|c|c}
                        id & select\_type & table & partitions & type & possible\_keys & key & key\_len & ref & rows & filtered & Extra \\
                        \hline
                        1 & PRIMARY & \textless derived2\textgreater & NULL & ALL & NULL & NULL & NULL & NULL & 2402 & 100.00 & Using temporary; Using filesort \\
                        2 & DERIVED & c & NULL & ALL & PRIMARY & NULL & NULL & NULL & 103056 & 2.33 & Using where; Using temporary; Using filesort \\
                        3 & SUBQUERY & caltrans & NULL & ALL & NULL & NULL & NULL & NULL & 103056 & 2.33 & Using where
                \end{tabular}
                }
                \hspace*{-7cm}
        \end{center}
\end{enumerate}

It seems that the last three queries performed worse than the first two queries, as they have more operations in the evaluation plan. The latter queries generate
multiple intermediate tables in order to join or filter the data, which is less efficient than the first two queries. As noted in the
solutions to the previous homework, the first two queries took the least amount of time, so they should be the most efficient. This matches the intuition
that performing fewer operations in the evaluation plan results in a more efficient query. When comparing the efficiency of the first query to the second,
note that the first filters 20.99 percent of 103056 rows, which is about 20 thousand. The second filters 2.33 percent of 103056 rows in one operation and
processes all 2402 rows in another without using a where clause. So it seems that the first query matches some duplicate rows before merging the distinct values
together. Generating the extra rows leads to a performance penalty that makes its performance comparable to the second query, even though the second query has
two operations rather than only one. Note that in this analysis I take the runtime of the query as a measure of efficiency, but I could also measure efficiency
through the space required for a query. By generating fewer intermediate tables, the first two queries should take up less space in memory than the remaining queries.

\section*{Part 3}

\subsection*{Problem 1}

Relation \(R\) will be scanned once for each iteration of the outer loop, so it will be scanned 100,000 times.

\subsection*{Problem 2}

\paragraph{a)}

Each tuple in \(L\) is scanned once in the outer loop. Let \(\bar{X}\) be the average size of the result of the index lookup. Then there are \(|L|\bar{X}\)
output tuples, and so a given tuple in \(R\) is scanned on average \(\frac{|L|\bar{X}}{|R|}\) times.

\paragraph{b)}

We are given that our join uses an equality, since the join condition is \(t_l.K=t_r.K\). The worst case scenario is that the B\textsuperscript{+}-tree on
\(R\) is a secondary index, and that \(K\) is an attribute that does not form a key for \(R\). Let \(\bar{X}\) be the average size of the result of the index lookup.
Let \(b_l\) be the blocks required to hold the relation \(L\). Let \(h\) be the height of the B\textsuperscript{+}-tree on \(R\). We have \(b_l\) block
transfers in order to read \(L\), \(h\) block transfers each loop iteration when finding the first record in the index lookup, and \(\bar{X}\) block transfers
each loop iteration to find the rest of the index lookup. So in total we have \(b_l+|L|(h+\bar{X})\) block transfers.

\subsection*{Problem 3}

Consider joining two relations \(R \bowtie S\), where \(b_r\) is the number of blocks necessary to store relation \(R\) and \(b_s\) is the number of
blocks necessary to store relation \(S\). Generally, block nested-loop join is strictly better than naive nested-loop join, as both have the best
case performance of 2 seeks and \(b_r+b_s\) block transfers when one relation of the join fits in memory. If neither relation of the join fits in memory,
the worst case performance is \(b_r+b_rb_s\) block transfers and \(2b_r\) disk seeks for block nested-loop join, whereas the worst case performance
of naive nested-loop join has \(b_r+|R|b_s\) block transfers and \(b_r + |R|\) disk seeks. Since \(|R|>b_r\), naive nested-loop join will never perform
better than block nested-loop join, and we would never want to use it.

When comparing block nested-loop join and indexed nested-loop join, the performance of the indexed nested-loop join depends on what type of index lookup
is being performed. Let the number of seeks in an average index lookup on the inner relation be \(s_l\) and the number of block transfers in an average index lookup 
on the inner relation be \(b_l\). Then the worst case performance of indexed nested-loop join is \(b_r+|R|b_l\) block transfers and \(b_r+|R|s_l\) seeks. The
number of seeks in indexed nested-loop join is much higher than for the other nested-loop join methods because it does not necessarily read the inner relation
in sequential order. So we would want to use indexed nested-loop join in cases where \(|R|b_l\) is much smaller than \(b_rb_s\), enough to offset the time to
perform additional seeks. If we let the time to perform a block transfer to be \(t_b\) and the time for a seek to be \(t_s\), we would want to use an indexed
nested-loop join when the following condition holds.
\[|R|(t_bb_l+t_ss_l) < b_r(t_bb_s+t_s)\]

The types of index lookups that would most likely satisfy this condition are a lookup on a primary or secondary B\textsuperscript{+}-tree index with the
join condition being an equality on a key, a lookup on a primary B\textsuperscript{+}-tree index with the join condition being an equality on a non-key,
and a lookup on a primary B\textsuperscript{+}-tree index with the join condition being a comparison. The first of these types of lookups have a number of seeks
and block transfers that scale linearly with the height of the tree, since only one record can be retrieved. So in that case it would be good to use indexed nested-loop
join. The remaining two lookups have a number of seeks and block transfers that additionally scale linearly with the number of blocks \(b\) necessary to contain the
retrieved records. If \(b\) is  large, it would be better to use block nested-loop join rather than indexed nested-loop join. But it may be better to use an indexed nested-loop
join in some cases. This would depend on the data that you are joining. Other types of index lookups such as a comparison with a secondary index may scale badly
with the number of records retrieved, so we would prefer to use block nested-loop join if index lookups are too costly.

\subsection*{Problem 4}

We can prove the two expressions are equivalent by showing that if there is some tuple \(\alpha\) in \(E_1 \bowtie_\theta (E_2-E_3)\),
then it must be in \((E_1\bowtie_\theta E_2 - E_1 \bowtie_\theta E_3)\) and vice versa. Consider the following.
\[\alpha \in E_1 \bowtie_\theta (E_2-E_3)\]
We know that \(E_2\) and \(E_3\) contain the same set of attributes \(S\) since we can apply set difference on them. Then we know that \(\alpha\) also
contains the set of attributes \(S\). These attributes come from a tuple \(t\) in \(E_2\) that is not in \(E_3\), such that \(t\) and some other tuple \(u\)
in \(E_1\) satisfies the join condition \(\theta\). Because \(t\) is not in \(E_3\), \(\alpha\) cannot be in \(E_1\bowtie_\theta E_3\). Additionally, \(\alpha\)
must be in \(E_1\bowtie_\theta E_2\) because we have already established that there exists some \(t\in E_2\) and \(u\in E_1\) that satisfy \(\theta\) and form \(\alpha\).
Thus if \(\alpha\in E_1 \bowtie_\theta (E_2-E_3)\), it must be true that \(\alpha\in (E_1\bowtie_\theta E_2 - E_1 \bowtie_\theta E_3)\).

In the other direction, consider some \(\alpha\in (E_1\bowtie_\theta E_2 - E_1 \bowtie_\theta E_3)\). Then \(\alpha\in E_1\bowtie_\theta E_2\) and
\(\alpha\notin E_1\bowtie_\theta E_3\). If alpha consists of some tuple \(t\) in \(E_2\) joined with a tuple \(u\) in \(E_1\), then this means either
\(t\notin E_3\) or \(u\notin E_1\) in order for \(\alpha\notin E_1\bowtie_\theta E_3\). Since we have already established that \(u\) is in \(E_1\), this means
that \(t\notin E_3\). Thus \(t\in E_2-E_3\), and \(\alpha \in E_1 \bowtie_\theta (E_2-E_3)\). Thus these two statements are equivalent.

This rule can be used to improve efficiency of queries by reducing the number of operations necessary to evaluate them. Notice that on the left side of the equation there
are only two operations being performed, while there on right side there are three. So converting queries that are in the form on the right into the form
on the left will strictly reduce the amount of work necessary to evaluate the query. Instead of doing two joins and one set difference, only one set difference and
one join is necessary.

\subsection*{Problem 5}

Consider the relations defined by the following attributes.
\begin{align*}
        R &= (A,B)\\
        S &= (A,B)
\end{align*}
Let \((1,1)\in R\) and \((1,2)\in S\). Then \(R-S\) is \(\{(1,1)\}\) and the expression \(\Pi_A(R-S)\) is \(\{(1)\}\). But we also have
\(\Pi_A(R) = \{(1)\}\) and \(\Pi_A(S) = \{(1)\}\). Thus \(\Pi_A(R) - \Pi_A(S) = \{\}\) which is not equal to \(\Pi_A(R-S)\).

\end{document}