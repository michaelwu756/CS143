\documentclass[12pt]{article}
\begin{document}
\title{Computer Science 143, Homework 2}
\date{May 1st, 2018}
\author{Michael Wu\\UID: 404751542}
\maketitle

\section*{Problem 1}

\paragraph{a)}

I ran the following query
\begin{verbatim}
SELECT highway, area
FROM caltrans
WHERE text LIKE '%closed%'
  AND (text LIKE '%for the winter%'
    OR text LIKE '%due to snow%')
GROUP BY highway, area
ORDER BY highway DESC, area DESC
LIMIT 20;
\end{verbatim}
and got the result set shown below.
\begin{center}
        \hspace*{-4cm}
        \begin{tabular}{c|c}
                highway & area\\
                \hline
                US395 & IN THE CENTRAL CALIFORNIA AREA \& SIERRA NEVADA\\
                SR89 & IN THE NORTHERN CALIFORNIA AREA \& SIERRA NEVADA\\
                SR89 & IN THE CENTRAL CALIFORNIA AREA \& SIERRA NEVADA\\
                SR88 & IN THE CENTRAL CALIFORNIA \& SIERRA NEVADA\\
                SR4 & IN THE CENTRAL CALIFORNIA AREA\\
                SR38 & IN THE SOUTHERN CALIFORNIA AREA\\
                SR330 & IN THE SOUTHERN CALIFORNIA AREA\\
                SR33 & IN THE SOUTHERN CALIFORNIA AREA\\
                SR3 & IN THE NORTHERN CALIFORNIA AREA\\
                SR270 & IN THE CENTRAL CALIFORNIA AREA \& SIERRA NEVADA\\
                SR267 & IN THE NORTHERN CALIFORNIA AREA\\
                SR203 & IN THE CENTRAL CALIFORNIA AREA \& SIERRA NEVADA\\
                SR20 & IN THE NORTHERN CALIFORNIA AREA\\
                SR2 & IN THE SOUTHERN CALIFORNIA AREA\\
                SR18 & IN THE SOUTHERN CALIFORNIA AREA\\
                SR172 & IN THE NORTHERN CALIFORNIA AREA\\
                SR168 & IN THE CENTRAL CALIFORNIA AREA \& SIERRA NEVADA\\
                SR158 & IN THE CENTRAL CALIFORNIA AREA \& SIERRA NEVADA\\
                SR138 & IN THE SOUTHERN CALIFORNIA AREA\\
                SR130 & IN THE CENTRAL CALIFORNIA AREA
        \end{tabular}
        \hspace*{-4cm}
\end{center}

\paragraph{b)}

I ran the following query
\begin{verbatim}
SELECT a.highway, a.area, closedPct
FROM
  (SELECT highway, area
   FROM caltrans
   WHERE text LIKE '%closed%'
     AND (text LIKE '%for the winter%'
       OR text LIKE '%due to snow%')
   GROUP BY highway, area
   ORDER BY highway DESC, area DESC)
  AS a
JOIN
  (SELECT highway, area, count(*)*100/365 AS closedPct
   FROM
     (SELECT DATE(reported), highway, area
      FROM caltrans
      WHERE text LIKE '%closed%'
      GROUP BY highway, area, DATE(reported))
     AS closedDays
   GROUP BY highway, area)
  AS closure
ON a.highway = closure.highway
  AND a.area = closure.area
ORDER BY closedPct DESC
LIMIT 5;
\end{verbatim}
and got the following result set.
\begin{center}
        \hspace*{-4cm}
        \begin{tabular}{c|c|c}
                highway & area & closedPct\\
                \hline
                SR120 & IN THE CENTRAL CALIFORNIA AREA \& SIERRA NEVADA & 66.5753\\
                SR89 & IN THE NORTHERN CALIFORNIA AREA \& SIERRA NEVADA & 66.5753\\
                SR4 & IN THE CENTRAL CALIFORNIA AREA & 63.5616\\
                SR203 & IN THE CENTRAL CALIFORNIA AREA \& SIERRA NEVADA & 61.3699\\
                SR108 & IN THE CENTRAL CALIFORNIA AREA \& SIERRA NEVADA & 55.6164
        \end{tabular}
        \hspace*{-4cm}
\end{center}

\section*{Problem 2}

\paragraph{a)}

A natural join will result in a cross join when there are no common attributes. Thus the venn diagram
should have their areas overlap. In fact, any equi join with no join key will result in a cross join,
so cross join should be a subset of equi join. The same can be said about non equi join, since that can
result in a cross join as well. Overall the relationship between different types of joins is fairly
complex and cannot be expressed in a nice form using a venn diagram, as there are too many types of joins.
Venn diagrams are usually only good if there are two sets being compared.

\section*{Problem 3}

\paragraph{a)}

I ran the following query
\begin{verbatim}
SELECT trip_starts.trip_id, trip_starts.user_id,
  IF(
    ISNULL(trip_ends.time),
    '24:00:00',
    SEC_TO_TIME(
      TIMESTAMPDIFF(SECOND, trip_starts.time, trip_ends.time)
    )
  ) AS trip_length
FROM trip_starts
LEFT JOIN trip_ends
ON trip_starts.trip_id=trip_ends.trip_id
LIMIT 5;
\end{verbatim}
and got the following result.
\begin{center}
        \begin{tabular}{c|c|c}
                trip\_id & user\_id & trip\_length\\
                \hline
                0 & 20685 & 00:01:12\\
                2 & 34808 & 00:02:59\\
                3 & 25463 & 24:00:00\\
                4 & 26965 & 00:01:34\\
                5 & 836 & 00:00:51
        \end{tabular}
\end{center}

\paragraph{b)}

I ran the following query
\begin{verbatim}
SELECT trip_starts.trip_id, trip_starts.user_id,
  IF(
    ISNULL(trip_ends.time),
    217.00,
    1+0.15*CEILING(
      TIMESTAMPDIFF(SECOND, trip_starts.time, trip_ends.time)
      /60
    )
  ) AS trip_charge
FROM trip_starts
LEFT JOIN trip_ends
ON trip_starts.trip_id=trip_ends.trip_id
LIMIT 5;
\end{verbatim}
and got the following results
\begin{center}
        \begin{tabular}{c|c|c}
                trip\_id & user\_id & trip\_charge\\
                \hline
                0 & 20685 & 1.30\\
                2 & 34808 & 1.45\\
                3 & 25463 & 217.00\\
                4 & 26965 & 1.30\\
                5 & 836 & 1.15
        \end{tabular}
\end{center}

\paragraph{c)}

I ran the following query
\begin{verbatim}
SELECT trip_starts.user_id,
  SUM(IF(
    ISNULL(trip_ends.time),
    217.00,
    1+0.15*CEILING(
      TIMESTAMPDIFF(SECOND, trip_starts.time, trip_ends.time)
      /60
    )
  )) AS monthly_total
FROM trip_starts
LEFT JOIN trip_ends
ON trip_starts.trip_id=trip_ends.trip_id
WHERE MONTH(trip_starts.time) = 3
GROUP BY trip_starts.user_id
LIMIT 5;
\end{verbatim}
and got the following result.
\begin{center}
        \begin{tabular}{c|c}
                user\_id & monthly\_total\\
                \hline
                0 & 222.50\\
                1 & 4.05\\
                2 & 665.05\\
                3 & 11.90\\
                4 & 444.55
        \end{tabular}
\end{center}
The user with \(\texttt{user\_id}=2\) owes \(\$665.05\) for the month of March.

\paragraph{d)}

You would have to use a self left join on the trip id, and then use a where clause to only choose rows
where the left side is a start entry and the right side is an end entry.

\end{document}